\documentclass{amsart}
\usepackage{geometry}
\usepackage{amsmath}
\usepackage{amssymb,amsfonts}
\usepackage{graphicx}
\usepackage{multicol}
\usepackage{caption}
\usepackage{subcaption}
\usepackage{enumerate}
\usepackage{placeins}
\usepackage{float}
\usepackage{wrapfig}
\restylefloat{table}
\newcommand{\nc}{\newcommand}
\newcommand{\tab}{\hspace*{7em}}
\newcommand{\conj}{\overline}
\newcommand{\dd}{\partial}
\nc{\ep}{\epsilon}
\title{Zeeman effect}
\author{Haoru Liu}
\begin{document}
\maketitle
\centering{Lab partner: Xu Xiao}
\tableofcontents
\section{Procedure}
This lab was conducted primarily in four sections: alignment of components and adjustment of the Fabry-Perot interferometer, measurement of spectral line splitting viewed perpendicular to $B$, measurement of spectral line splitting viewed parallel to $B$, and measuring the actual value of $|B|$ for each value we used. 
\subsection{Alignment and adjustment of components}
The components used in this lab, listed in the order in which light travels through them, are the Hg lamp suspended in an iron-core electromagnet, a collimator(located inside the lamp housing), a quarter-wave plate, a linear polarizer, a Fabry-Perot interferometer, a focusing lens, and a CCD camera connected to a computer. For the first part of the lab, we oriented the lamp (which was on a rotating platform) so that the light traveling down the optical bench would be perpendicular to the field generated by the electromagnet. All components were then removed from the bench except for the Fabry-Perot interferometer, and the lamp was lit by a high-voltage discharge. The interferometer was then adjusted by means of three knobs which varied the spacing between the plates of the interferometer. Good spacing was determined by visual inspection -- we aimed to adjust the knobs so that the image of the lamp seen through the interferometer does not change when moving our head across any diameter of the interferometer. 

Next, we placed all the components on the bench in order and aligned them (excluding the quarter-wave plate, which is only used in the third part of the lab). To do so, we first noted that all components (excluding the polarizers) had a marking depicting where they were centered optically. We then used a horizontally oriented rod whose height was set to the height of the center of the lamp as a reference, and adjusted the heights of the rest of the components to match. The linear polarizer was then rotated to an approximately horizontal position and placed in front of the interferometer. 

All components now in place, the image capture program was started on the computer and set to capture rapidly. Keeping the focusing lens fixed, the CCD camera was moved so that the image sent to the computer was in focus. Once the image was in focus, the current to the electromagnet was turned on to an arbitrary value, and the linear polarizer was placed in front of the interferometer in an approximately horizontal position. The polarizer was then rotated so that only $3$ rings were visible (the faint, barely visible rings are due to the $\Delta m=\pm1$ transitions), and the knobs on the interferometer adjusted to make for a clear image. We then proceeded to taking data.
\subsection{Measuring the splitting for $\Delta m=0$ transitions viewed perpendicular to $B$}
The dial on the power source for the electromagnet was turned down to zero in order to completely power down the magnet and avoid hysteresis effects, then gradually increased until splitting became visible in the captured image. The voltage across a $0.1\Omega$ resistor in series with the electromagnet was measured and taken to be representative of the magnitude of the field that the lamp was experiencing, and the image was recorded on the computer by adjusting the gain and exposure time to produce a clean image. The power to the electromagnet was then increased until the outer division of the 1st order ring intersected with the inner division of the 2nd order ring (at which point the diffraction orders became indistinct), at which point the voltage and image were recorded. Then, images were recorded in association with evenly spaced voltages through the range in which splitting was easily observable.
\subsection{Measuring the intensities for $\Delta m=\pm1$ transitions viewed parallel to $B$}
For this part, the light blocker to one side of the lamp was removed and a collimator inserted in its place. The lamp assembly was rotated so that the light from the newly inserted collimator traveled down the optical bench. The quarter-wave plate was then placed between the linear polarizer and the lamp. Since the waveplate lacked markings to indicate the polarization of transmitted light, we estimated the orientation of the waveplate which would convert circularly polarized light to horizontal or vertically polarized light. Since we expect the light exiting the waveplate-polarizer system to be symmetric across the correct orientation of the waveplate, we determined the true orientation to be at the $55^\circ$ mark on the waveplate due to observation of said symmetry. Images were then taken for two voltages within the previously determined range of good voltages, with orientation of the waveplate at $55^\circ$ and $145^\circ$ for each.
\subsection{Measuring the value of $|B|$}
The lamp assembly was separated from the electromagnet and set aside. A Hall effect gaussmeter was used on the $<9999$ gauss setting to measure the magnitude of $B$ for each voltage we previously used. The uncertainty in this value was obtained by moving the probe vertically through the region in which the lamp resided and recording the range of values that the gaussmeter displayed. The gaussmeter probe was then inserted into three reference magnets for which the field intensity is known exactly, and the reference intensities were recorded alongside the intensities displayed by the gaussmeter. This was done to calibrate the gaussmeter as it may have some systematic bias to its readings. 
\section{Data and Analysis}
\begin{wrapfigure}{r}{6cm}
\vspace{-10pt}
\centering
\includegraphics[width=5.5cm]{dm0/b225.png}
\caption{Image captured at 0.225V with 0.45dB gain and 5.048s exposure time}
\label{rings}
\vspace{-60pt}
\end{wrapfigure}
Our raw data were primarily gathered in the form of images captured with the CCD camera, a representative example of which can be seen in Figure \ref{rings}. In order to convert these data into radii of rings as a function of magnetic field and diffraction order, we had to do some processing. 

First, for each image, we used the Plot Profile tool in ImageJ to extract the intensities along a diameter of all rings. The position of this line was determined by visual approximation, but it should not affect our results much because the length of a chord close to the diameter is very close to the length of the diameter itself. The intensity data were saved in text format and imported into Python for plotting. Due to the lack of a good automated solution for finding peaks reliably, the radii of the rings were determined by visual detection of a peaks in a matplotlib plot of intensities versus position. To find the radii from the peak positions, we took the difference between the two peak positions of each ring and divided by $2$, since the true center of the rings is difficult to determine. The exception was for the image taken at $0.103$V: there, the peaks to the left of the center became increasingly difficult to distinguish past the second set of rings, so we found the center by averaging the positions of the peaks of the innermost ring and subtracted from peak positions on the right. Accordingly, we had a greater amount of uncertainty for that value of voltage. In addition, the radii were found in units of pixels, as we assume that there the number of pixels is directly proportional to physical length. In the course of our later analysis, the proportionality factor will be cancelled out.

\begin{figure}
\centering
\includegraphics[width=16cm]{dm0/plot225.png}
\caption{Horizontal intensity profile of the image in Figure \ref{rings}}
\label{peaks}
\end{figure}

Despite the seemingly crude way of finding the peak positions, there was fairly little uncertainty in locating the peaks due to the flexible resolution that the matplotlib interface presents. The results of this method were satisfactory, as further analysis will demonstrate.

After the peaks from each image were extracted and categorized in the appropriate wavelength (a, b, or c, depending on whether it was the inner, middle, or outer split), we fit the data to a linear function for each wavelength in each image. The equation used was 
\begin{equation}
\label{rad_lin}
r_p^2=\frac{2f^2}{n_0}p+\frac{2f^2}{n_0}(\epsilon-1)\text{(eqn 20 {in} the lab manual),}
\end{equation}
where $r_p^2$ is the radius squared of the $p$th ring of each wavelength and $\epsilon$ is a linear function of the wavenumber. The other terms, which denote the focal length of the lens and the order of the central fringe, are not important to our analysis, since we can obtain $\epsilon$ simply by dividing the $y$-intercept by the slope in the fitted equation. Note that this is where the proportionality factor in the radii drops out. Denote $\epsilon$ for wavelengths a, b, c as $\epsilon_a,\epsilon_b,\epsilon_c$ respectively. Then, we note that the $y$-intercept for $\epsilon_i$ as a function of wavenumber is constant across the wavelengths, so we get that the difference in wavenumber is $k_i-k_j=\frac{1}{2D}(\epsilon_i-\epsilon_j)$, so $\Delta E_{ij}=\frac{h c}{2D}(\epsilon_i-\epsilon_j)$, where $D$ is the spacing between the plates of the interferometer (given as $0.990\pm0.001$cm). 

\begin{wrapfigure}{l}{200pt}
\vspace{-15pt}
\centering
\includegraphics[width=200pt]{dm0/225-2fit.png}
\caption{Fitted line for the b ring of the image in Figure \ref{rings}}
\label{line}
\vspace{-100pt}
\end{wrapfigure}

Uncertainties in the measurements of the peak positions were usually close to $0.8$px. The analysis was done on a screen with higher resolution than the original image, so one pixel on our screen corresponded to slightly less than one pixel on the image. However, there were some peaks for which the profile was less distinct (particularly the innermost rings), so the uncertainty was higher for those. Uncertainties for the radii ($\delta r_p$) were calculated by adding the uncertainties for the peak positions in quadrature, and we took $\delta r_p^2$ to be $2r_p\delta r_p$. Since $p$ in equation \ref{rad_lin} is integer-valued, we take it to be exact. Let $\delta A$ be the uncertainty in the slope and $\delta B$ be the uncertainty in the $y$-intercept (as provided by the fit). Then, $\delta\epsilon=\epsilon\sqrt{\big(\frac{\delta B}{B}\big)^2+\big(\frac{\delta A}{A}\big)^2}$, so $$\delta (\Delta E_{ij})=\frac{h c(\epsilon_i-\epsilon_j)}{2D}\sqrt{\frac{(\delta\ep_i)^2+(\delta\ep_j)^2}{(\ep_i-\ep_j)^2}+\frac{(\delta D)^2}{D^2}}\text{.}$$ 

Now, we need to calculate the values of $B$ from the voltages. For this, we assumed that the magnetic field produced by the electromagnet is a linear function of current and that the value produced by the gaussmeter is a linear function of the true magnetic field. Thus, we did two fits -- one by considering the gaussmeter reading as a function of voltage, and another by considering the gaussmeter reading as a function of the reference magnet strength. For these fits, we ignored the uncertainty in the voltmeter reading, since it is fairly negligible compared to the gaussmeter reading ($1$\% versus $5$\%). Instead, we will use the uncertainty in voltage when making the final calculations for the value of $B$ for each voltage.

\begin{wraptable}{r}{21em}
\vspace{-20pt}
\begin{tabular}[t]{|l|l|}
\hline
\multicolumn{2}{|c|}{Voltage vs. gaussmeter}\\
\hline
Voltage(V) ($\pm$ 0.001)& Gaussmeter(G)\\
0.103 & 2500$\pm$100\\
0.146 & 3400$\pm$100\\
0.183 & 4300$\pm$200\\
0.225 & 5800$\pm$250\\
0.266 & 6200$\pm$250\\
0.307 & 7200$\pm$200\\
0.346 & 7900$\pm$300\\
0.380 & 8700$\pm$300\\
\hline
\multicolumn{2}{|c|}{Gaussmeter vs. reference}\\
\hline
Gaussmeter(G) & Reference(G)\\
266$\pm$3 & 299\\
950$\pm$1 & 989\\
2899$\pm$1 & 2920\\
\hline
\end{tabular}
\caption{Calibration data for the gaussmeter}
\end{wraptable}

Fitting the two data sets to affine equations and composing the resulting equations, we obtain $B_{\text{actual}}=(22700\pm700)\cdot V+(200\pm130)$, with a reduced $\chi^2$ of $0.891$ in the voltage-gaussmeter fit and $14.1$ in the reference fit. We can then use this to associate values of $B$ to the measured voltages. Combining this with the calculated values of $\Delta E_{ij}$ from earlier, we summarize the data in the first 4 columns of Table \ref{biglist}.


We are mostly concerned with the values of $\Delta E_{ab}$ and $\Delta E_{cb}$, as these are the values which are predicted by quantum mechanics. From the theory contained in the lab manual, the energy of a state is $E=E_{n,l}+\mu_Bgm_jB$, where $E_{n,l}$ is the degenerate energy eigenvalue, $\mu_B$ is a constant known as the Bohr magneton, $g$ is a term known as the Land\'e-g factor, and $m_j$ is the third quantum number of the state. For the transitions we consider here, the transition is from a state with $g=2$ to a state with $g=3/2$ and with $\Delta m_j=0$. Thus, if we let $E_0$ be the transition energy with no magnetic field, we have that $E_a=E_0-\frac{1}{2}\mu_BB$, $E_b=E_0$, and $E_c=E_0+\frac{1}{2}\mu_BB$. Thus, the energy split between the different rings should be of magnitude $\frac{1}{2}\mu_BB$. Using this formula and our values of $B$, we can find the theoretically predicted values of $\Delta E$, located in the last column of Table \ref{biglist}.

\begin{table}[H]
\begin{tabular}[c]{|c|c|c|c|c|}
\hline
Voltage(V) & $B_\text{actual}$(G) & $|\Delta E_{ab}|(\mu \text{eV})$ & $|\Delta E_{bc}|(\mu \text{eV})$ & $\Delta E_\text{expected}(\mu$eV)\\
\hline
0.103 & $2500\pm150$ & $6\pm5$ & $10\pm12$ & $7.3\pm0.42$\\
0.145 & $3500\pm160$ & $9.3\pm0.88$ & $11\pm2$ & $10.0\pm0.47$\\
0.183 & $4300\pm180$ & $12.4\pm0.68$ & $13\pm8$ & $12.5\pm0.51$\\
0.225 & $5300\pm200$ & $16.7\pm0.58$ & $16\pm9$ & $15.3\pm0.57$\\
0.266 & $6200\pm220$ & $18.7\pm0.71$ & $20\pm11$ & $18.0\pm0.63$\\
0.306 & $7100\pm240$ & $21\pm1$ & $21\pm4$ & $20.6\pm0.7$\\
0.345 & $8000\pm260$ & $24\pm1$ & $25\pm3$ & $23.2\pm0.76$\\
0.380 & $8800\pm280$ & $26.4\pm0.8$ & $27\pm6$ & $25.5\pm0.82$\\
\hline
\end{tabular}
\caption{Results of the analysis of $\Delta m=0$ images}
\label{biglist}
\end{table}

Now, we wish to determine a value for the Bohr magneton from the data we have. This can be done by noting that $\Delta E_{ij}$ is a linear function of $B$, with the Bohr magneton being twice the slope. Since our values of $\Delta E_{ab}$ seem to be more precise in general, we will fit these values to $B$. However, we are unable to take the uncertainty of $B$ into account, as the fitting methods we have only allow for errors in the dependent variable to be taken into account. Doing the fit this way, we obtain the following fit. The reduced $\chi^2$ is 0.667, and we obtain a value of $0.0063\pm0.0003 \mu\text{eV}/\text{G}$ for $\mu_B$. From Wolfram Alpha, the value of the Bohr magneton is 0.005788.

\begin{figure}
\centering
\includegraphics[width=300pt]{dm0/bohr.png}
\caption{Fit of energy splitting versus $B$ for the ab split}
\label{bohr}
\end{figure}

\begin{figure}
\centering
\begin{subfigure}[b]{0.45\textwidth}
\centering
\includegraphics[width=\textwidth]{dm1/b225-55.png}
\caption{Image at $55^\circ$ orientation of the waveplate}
\end{subfigure}
\quad
\begin{subfigure}[b]{0.45\textwidth}
\centering
\includegraphics[width=\textwidth]{dm1/b225-145.png}
\caption{Image at $145^\circ$ orientation of the waveplate}
\end{subfigure}
\caption{Images captured with $\Delta m=\pm1$}
\label{rings1}
\end{figure}

Now, we examine the $\Delta m=\pm1$ case. The images we will be working with are located in Figure \ref{rings1}. For the first set of rings in the $145^\circ$ case, we observe intensities of 246, 471, and 587 for rings a, b, and c, respectively. Taking into account the baseline intensity of 96, we have intensity values of 150, 375, and 491. For the second set of rings, again taking the baseline into account, we have 121, 296, and 412. For the $55^\circ$ case, the first set is 144, 95, 37, and the second set is 129, 80, 32, in the same order. This was again determined mostly by visual inspection, with the matplotlib interface reporting the position of the mouse on the graph. 
\section{Discussion}
Looking back at Table \ref{biglist}, our values for $\Delta E$ seem to agree with the expected values quite well. The primary contribution to the uncertainty in the values of $\Delta E$ was probably the variance of the points about the fit, as the individual errors in the peak positions were mostly on the order of $0.1\%$. The large amount of uncertainty in the 2500G case was to be expected, as the individual peaks were barely resolvable in the intensity profile of that image. A notable trend is that the uncertainties for the bc energy split are a good deal higher than the uncertainties for the ab split. Through further analysis, we determined that this was due to a persistent trend in the uncertainties of $\epsilon_c$, which in turn were derived from uncertainties in the fitted parameters of the c rings. If we output the uncertainties for the individual fitted parameters in the c ring of the 5300G case(one of the more egregious offenders), we see that they're of similar value as the uncertainties for other cases (~60 for slope and ~200 for y-intercept). The difference lies in the value of the parameters. In the fitted equation, we expect the slope to be roughly constant, as it has no dependence on $\epsilon$. For $\epsilon$ far enough away from $1$, the value of the $y$-intercept will a large negative value, on the order of $-30000$ for $\epsilon=0.5$. However, as $\epsilon$ approaches $1$, the value of the $y$-intercept gets small with no corresponding decrease in the uncertainty. In the c ring of the 5300G case, the fitted y-intercept was 1300, which presents a high relative error. Thus, the fractional uncertainty ends up large, which becomes reflected in the final result when dividing. 

Another thing to remark on is the low reduced $\chi^2$ values that our fits provided. While ours were not exceedingly low (nothing $<0.1$), low values of reduced $\chi^2$ can indicate overestimation of the uncertainty in the data. However, we find that to be unlikely. Our uncertainties are at the lower bound set by technological limitations -- most of the peak positions were resolved down to the pixel on the screen used for analysis. Another reduced $\chi^2$ value to note is the one returned by the gaussmeter-reference fit. The high value of this was not surprising, since we only had three data points to use, which results in only one degree of freedom.

Overall, ignoring the sensitivity of $\epsilon$ to fluctuations near 1, our data agrees quite well with the predictions made by quantum mechanics. If we examine the values in Table \ref{biglist}, we see that all but 1 of the 16 data points we produced lie within error of expected. If we assume that values are normally distributed with the error as the standard deviation, this is lower than the expected 3 points which should lie outside of 1 standard deviation. As above, it implies that we overestimated uncertainty, but this could be a result of the anomaly in fitting discussed in the beginning of this discussion.

When we calculate the value of the Bohr magneton from a fit of our data, we see that it departs from the literature value by about $2$ standard deviations. Normally, this would be a cause for concern, but if we look at the plot, we see that the relative uncertainties in $B$ which we did not take into account are on the same order of magnitude as those in $\Delta E$. This means that the uncertainty reported by the fit underestimates the actual uncertainty in the slope. Lacking the appropriate statistical knowledge, we are unable to state exactly how far the underestimation is, but it may be enough to put us within uncertainty of the literature value.

The $\Delta m=\pm1$ case was considerably less successful. Instead of the 6:3:1 ratios that theory predicts, we saw something more like a 3:2:1 ratio. Because we were unable to determine the orientation of the axes of the waveplate, we were also unable to determine which angle corresponded to the $\Delta m=1$ transition and which corresponded to the other. Qualitatively, however, we did see a distinct difference. The intensities in the $\Delta m=0$ case were arranged as medium:high:medium for rings a,b,c, respectively, while the intensities in the $\Delta m=\pm1$ case were arranged as low:medium:high.
\section{Appendix}
\FloatBarrier
\subsection{Tables of radii with uncertainties}
.

\begin{tabular}{|c|c|}
\hline
\multicolumn{2}{|c|}{Radii for V=225.000}\\
\hline
Ring a & $189\pm3, 329\pm1, 425\pm1, 503\pm1, 570\pm1, 631\pm1, 686\pm1, 737\pm1, 783\pm1, 829\pm1, 871.5\pm0.71$\\
Ring b & $233\pm2, 356\pm1, 446\pm1, 521\pm1, 587\pm1, 645\pm1, 699\pm1, 748\pm1, 795\pm1, 839\pm1, 881.0\pm0.71$\\
Ring c & $272\pm1, 381\pm1, 466\pm1, 538\pm1, 602\pm1, 659\pm1, 712\pm1, 761\pm1, 807\pm1, 851\pm1, 891.5\pm0.71$\\
\hline
\end{tabular}
\vspace{10pt}

\begin{tabular}{|c|c|}
\hline
\multicolumn{2}{|c|}{Radii for V=103.000}\\
\hline
Ring a & $210\pm6, 341\pm6, 435\pm6, 521\pm6$\\
Ring b & $227\pm8, 355\pm6, 445\pm8, 528\pm6$\\
Ring c & $245\pm8, 365\pm6, 453\pm6, 536\pm6$\\
\hline
\end{tabular}
\vspace{10pt}

\begin{tabular}{|c|c|}
\hline
\multicolumn{2}{|c|}{Radii for V=266.000}\\
\hline
Ring a & $183\pm1, 325\pm1, 421\pm1, 501\pm1, 568\pm1, 628\pm1$\\
Ring b & $234\pm2, 357\pm1, 447\pm1, 522\pm1, 587\pm1, 646\pm1$\\
Ring c & $277\pm1, 386\pm1, 470\pm1, 543\pm1, 606\pm1, 661\pm1$\\
\hline
\end{tabular}
\vspace{10pt}

\begin{tabular}{|c|c|}
\hline
\multicolumn{2}{|c|}{Radii for V=145.000}\\
\hline
Ring a & $206\pm2, 340\pm2, 433\pm1, 509\pm1, 575\pm1, 635\pm1, 689\pm1, 741\pm1, 788\pm1, 832\pm1$\\
Ring b & $230\pm1, 355\pm1, 446\pm1, 521\pm1, 586\pm1, 644\pm1, 698\pm1, 748\pm1, 795\pm1, 839\pm1$\\
Ring c & $256\pm1, 370\pm1, 458\pm1, 532\pm1, 596\pm1, 653\pm1, 707\pm1, 756\pm1, 802\pm1, 845\pm1$\\
\hline
\end{tabular}
\vspace{10pt}

\begin{tabular}{|c|c|}
\hline
\multicolumn{2}{|c|}{Radii for V=306.000}\\
\hline
Ring a & $178\pm2, 320\pm1, 419\pm1, 497\pm1, 566\pm1$\\
Ring b & $237\pm3, 357\pm1, 447\pm1, 522\pm1, 587\pm1$\\
Ring c & $284\pm1, 390\pm1, 474\pm1, 546\pm1, 609\pm1$\\
\hline
\end{tabular}
\vspace{10pt}

\begin{tabular}{|c|c|}
\hline
\multicolumn{2}{|c|}{Radii for V=183.000}\\
\hline
Ring a & $199\pm1, 333\pm1, 429\pm1, 506\pm1, 574\pm1, 633\pm1, 688\pm1, 739\pm1, 786\pm1, 830\pm1$\\
Ring b & $232\pm1, 355\pm1, 446\pm1, 520\pm1, 587\pm1, 645\pm1, 699\pm1, 748\pm1, 796\pm1, 839\pm1$\\
Ring c & $262\pm1, 375\pm1, 462\pm1, 535\pm1, 596\pm1, 656\pm1, 709\pm1, 759\pm1, 805\pm1, 848\pm1$\\
\hline
\end{tabular}
\vspace{10pt}

\begin{tabular}{|c|c|}
\hline
\multicolumn{2}{|c|}{Radii for V=345.000}\\
\hline
Ring a & $168\pm1, 317\pm1, 417\pm1, 495\pm1, 565\pm1$\\
Ring b & $236\pm1, 359\pm1, 448\pm1, 522\pm1, 589\pm1$\\
Ring c & $291\pm1, 397\pm1, 478\pm1, 550\pm2, 613\pm1$\\
\hline
\end{tabular}
\vspace{10pt}

\begin{tabular}{|c|c|}
\hline
\multicolumn{2}{|c|}{Radii for V=380.000}\\
\hline
Ring a & $146\pm3, 306\pm2, 409\pm1, 487\pm2$\\
Ring b & $228\pm2, 353\pm2, 443\pm1, 519\pm1$\\
Ring c & $289\pm1, 392\pm2, 477\pm1, 547\pm3$\\
\hline
\end{tabular}
\vspace{10pt}
\subsection{Plots of the $r^2$ vs. $p$ fits}
.

\begin{figure}
\centering
\includegraphics[width=400pt]{dm0/plots/225-rplot.png}
\end{figure}
\vspace{10pt}

\begin{figure}
\centering
\includegraphics[width=400pt]{dm0/plots/103-rplot.png}
\end{figure}
\vspace{10pt}

\begin{figure}
\centering
\includegraphics[width=400pt]{dm0/plots/266-rplot.png}
\end{figure}
\vspace{10pt}

\begin{figure}
\centering
\includegraphics[width=400pt]{dm0/plots/145-rplot.png}
\end{figure}
\vspace{10pt}

\begin{figure}
\centering
\includegraphics[width=400pt]{dm0/plots/306-rplot.png}
\end{figure}
\vspace{10pt}

\begin{figure}
\centering
\includegraphics[width=400pt]{dm0/plots/183-rplot.png}
\end{figure}
\vspace{10pt}

\begin{figure}
\centering
\includegraphics[width=400pt]{dm0/plots/345-rplot.png}
\end{figure}
\vspace{10pt}

\begin{figure}
\centering
\includegraphics[width=400pt]{dm0/plots/380-rplot.png}
\end{figure}
\vspace{10pt}
\FloatBarrier
\subsection{Fit of $\Delta E$ vs. B for the bc split}
.

\begin{figure}
\centering
\includegraphics[width=400pt]{dm0/bc.png}
\caption{Reduced $\chi^2$: 0.01\\$\Delta E=(0.00312\pm7.8\times10^{-5})\cdot B + (-0.4\pm0.47)$}
\end{figure}
\vspace{10pt}

\end{document}
